% Tipo de documento y opciones.
\documentclass[11pt, a4paper]{article}


%---------------------------------------------------------------------------
%   PAQUETES
%---------------------------------------------------------------------------

% Idioma y codificación para Español.
\usepackage[utf8]{inputenc}
\usepackage[spanish, es-tabla, es-lcroman, es-noquoting]{babel}
\selectlanguage{spanish} 
%\usepackage[T1]{fontenc}

% Fuente utilizada.
\usepackage{courier}    % Fuente Courier.
\usepackage{microtype}  % Mejora la letra final de cara al lector.

% Diseño de página.
\usepackage{fancyhdr}   % Utilizado para hacer títulos propios.
\usepackage{lastpage}   % Referencia a la última página.
\usepackage{extramarks} % Marcas extras. Utilizado en pie de página y cabecera.
\usepackage[parfill]{parskip}    % Crea una nueva línea entre párrafos.
\usepackage{geometry}            % Geometría de las páginas.

% Símbolos y matemáticas.
\usepackage{amssymb, amsmath, amsthm, amsfonts, amscd}
\usepackage{upgreek}

% Otros.
\usepackage{enumitem}   % Listas mejoradas.
\usepackage[hidelinks]{hyperref}
\usepackage{graphicx}   % Gráficos.


%---------------------------------------------------------------------------
%   OPCIONES PERSONALIZADAS
%---------------------------------------------------------------------------

% Redefinir letra griega épsilon.
\let\epsilon\upvarepsilon

% Formato de texto.
\linespread{1.1}            % Espaciado entre líneas.
\setlength\parindent{0pt}   % No indentar el texto por defecto.
\setlist{leftmargin=.5in}   % Indentación para las listas.

% Estilo de página.
\pagestyle{fancy}
\fancyhf{}
\geometry{left=3cm,right=3cm,top=3cm,bottom=3cm,headheight=1cm,headsep=0.5cm}   % Márgenes y cabecera.

% Ruta donde buscar gráficos
\graphicspath{{../Recursos/Plantillas/}, {Recursos/Plantillas/}}

% Redefinir entorno de demostración (reducir espacio superior)
%\makeatletter
%\renewenvironment{proof}[1][\proofname] {\vspace{-15pt}\par\pushQED{\qed}\normalfont\topsep6\p@\@plus6\p@\relax\trivlist\item[\hskip\labelsep\it#1\@addpunct{.}]\ignorespaces}{\popQED\endtrivlist\@endpefalse}
%\makeatother

% Aumentar el tamaño del interlineado
\linespread{1.3}


%---------------------------------------------------------------------------
%   COMANDOS PERSONALIZADOS
%---------------------------------------------------------------------------

% Valor absoluto: \abs{}
\providecommand{\abs}[1]{\lvert#1\rvert}    

% Fracción grande: \ddfrac{}{}
\newcommand\ddfrac[2]{\frac{\displaystyle #1}{\displaystyle #2}}

% Texto en negrita en modo matemática: \bm{}
\newcommand{\bm}[1]{\boldsymbol{#1}}

% Línea horizontal.
\newcommand{\horrule}[1]{\rule{\linewidth}{#1}}

% Letras de conjuntos
\newcommand{\R}{\mathbb{R}}
\newcommand{\N}{\mathbb{N}}

% Sucesiones
\newcommand{\xn}{\{x_n\}}
\newcommand{\fn}{\{f_n\}}


%---------------------------------------------------------------------------
%   CABECERA Y PIE DE PÁGINA
%---------------------------------------------------------------------------

% Cabecera del documento.
\renewcommand\headrule{
	\begin{minipage}{1\textwidth}
		\hrule width \hsize 
	\end{minipage}
}

% Texto de la cabecera.
\lhead{\subject}  % Izquierda.
\chead{}            % Centro.
\rhead{\docauthor}    % Derecha.

% Pie de página del documento.
\renewcommand\footrule{                                 
	\begin{minipage}{1\textwidth}
		\hrule width \hsize   
	\end{minipage}\par
}

% Texto del pie de página.
\lfoot{}                                                 % Izquierda
\cfoot{}                                                 % Centro.
\rfoot{Página\ \thepage\ de\ \protect\pageref{LastPage}} % Derecha.


%---------------------------------------------------------------------------
%   ENTORNOS PARA MATEMÁTICAS
%---------------------------------------------------------------------------

% Nuevo estilo para definiciones.
\newtheoremstyle{definition-style} % Nombre del estilo.
{10pt}               % Espacio por encima.
{10pt}               % Espacio por debajo.
{}                   % Fuente del cuerpo.
{}                   % Identación.
{\bf}                % Fuente para la cabecera.
{.}                  % Puntuación tras la cabecera.
{.5em}               % Espacio tras la cabecera.
{\thmname{#1}\thmnumber{ #2}\thmnote{ (#3)}}     % Especificación de la cabecera (actual: nombre en negrita).

% Nuevo estilo para notas.
\newtheoremstyle{remark-style} 
{10pt}                
{10pt}                
{}                   
{}                   
{\itshape}          
{.}                  
{.5em}               
{}                  

% Nuevo estilo para teoremas y proposiciones.
\newtheoremstyle{theorem-style}
{10pt}                
{10pt}                
{\itshape}           
{}                  
{\bf}             
{.}                
{.5em}               
{\thmname{#1}\thmnumber{ #2}\thmnote{ (#3)}}                   

% Nuevo estilo para ejemplos.
\newtheoremstyle{example-style}
{10pt}                
{10pt}                
{}                  
{}                   
{\scshape}              
{:}                 
{.5em}               
{}                   

% Teoremas, proposiciones y corolarios.
\theoremstyle{theorem-style}
\newtheorem{nth}{Teorema}[section]
\newtheorem*{nprop}{Proposición}
\newtheorem{ncor}{Corolario}[section]

% Definiciones, ejercicios.
\theoremstyle{definition-style}
\newtheorem*{ndef}{Definición}
\newtheorem{ejer}{Ejercicio}

% Notas, soluciones.
\theoremstyle{remark-style}
\newtheorem*{nota}{Nota}
\newtheorem*{sol}{Solución}

% Ejemplos.
\theoremstyle{example-style}
\newtheorem{ejemplo}{Ejemplo}[section]

% Listas ordenadas con números romanos (i), (ii), etc.
\newenvironment{nlist}
{\begin{enumerate}
\renewcommand\labelenumi{(\emph{\roman{enumi})}}}
{\end{enumerate}}

% División por casos con llave a la derecha.
\newenvironment{rcases}
  {\left.\begin{aligned}}
  {\end{aligned}\right\rbrace}


%---------------------------------------------------------------------------
%   PÁGINA DE TÍTULO
%---------------------------------------------------------------------------

% Título del documento.
\newcommand{\subject}{}

% Autor del documento.
\newcommand{\docauthor}{}

% Título
\title{
  \normalfont \normalsize 
  \textsc{} \\ [25pt]    % Texto por encima.
  \horrule{0.5pt} \\[0.4cm] % Línea horizontal fina.
  \huge Algoritmos Greedy \\ \Large Problema QAP\\ % Título.
  \horrule{2pt} \\[0.5cm] % Línea horizontal gruesa.
}

% Autor.
\author{\Large{Daniel Bolaños Martínez} 
\\\\
\Large{José María Borrás Serrano}
\\\\
\Large{Santiago de Diego de Diego}
\\\\
\Large{Fernando de la Hoz Moreno}}

% Fecha.
\date{\vspace{-1.5em} \normalsize }


%---------------------------------------------------------------------------
%   COMIENZO DEL DOCUMENTO
%---------------------------------------------------------------------------

\begin{document}

\maketitle  % Título.
\vfill
\begin{center}
\end{center}
\newpage
\newpage

\section{Análisis del problema}

El problema, P está basado en el Problema de Asignación Cuadrática,
consiste en asignar a cada oficinista de un grupo de oficinistas,
una habitación de un grupo de habitaciones de forma que se minimice
el coste de asignar a cada habitación i el oficinista p(i). El Problema
puede generalizarse a asignar N instalaciones a una cantidad N de
locaciones en donde se considera un costo asociado a cada una de las
asignaciones deseando que el coste en función del flujo sea mínimo.
Matemáticamente, el problema puede definirse como:

\[
p^{*}=min_{p}{H(p)}=min_{p}{\displaystyle \sum_{i=0}^{N-1}{\displaystyle \sum_{i=0}^{N-1}}f_{p(i)p(j)}d_{ij}}
\]
donde f es la matriz de flujos de intersección desde el oficinista
p(i) al p(j) y d es la matriz de distancias desde la habitación
i a la j.

Por ejemplo, la matriz de distancias d puede ser:

\[
\begin{array}{|ccccc|}
\hline 0 & 7 & 14 & 20 & 3\\
4 & 0 & 10 & 17 & 49\\
51 & 1 & 0 & 43 & 71\\
7 & 3 & 10 & 0 & 20\\
90 & 101 & 47 & 3 & 0
\\\hline \end{array}
\]

y la matriz de flujos:

\[
\begin{array}{|ccccc|}
\hline 0 & 4 & 7 & 4 & 1\\
0 & 0 & 10 & 3 & 21\\
0 & 0 & 0 & 47 & 3\\
41 & 21 & 7 & 0 & 9\\
21 & 43 & 32 & 0 & 27
\\\hline \end{array}
\]

Además definimos un vector de distancias entre habitaciones $d_p(i)=\displaystyle\sum_{j=0}^{N-1}d_{ij}$.
Cada componente del vector contiene la suma de las distancias desde
la localización i al resto.

En nuestro caso dicho vector vendria dado como:

\[
d_{p}=(44,80,166,40,241)
\]

Si realizamos lo mismo con la matriz de flujos f, tendremos el vector
:

\[
f_{p}={\displaystyle \sum_{b=0}^{N-1}f_{ab}}
\]

que en nuestro ejemplo particular es:

\[
f_{p}=(16,24,50,78,123)
\]


\section{Diseño de la solución relacionándolas con la metodología Greedy}

En un algoritmo voraz los pasos que tenemos que seguir son: 
\begin{itemize}
\item Diseñar una lista de candidatos 
\item Identificar una lista de candidatos ya utilizados 
\item Diseñar una función solución 
\item Diseñar un criterio de factibilidad
\item Diseñar una función de selección del candidato más prometedor para
formar parte de la solución 
\item Encontrar una función objetivo de minimización/maximización. 
\end{itemize}
Nosotros hemos implementado los pasos de la siguiente forma:
\begin{itemize}
\item El \textbf{conjunto C de candidatos}, en este caso, las distancias
entre las oficinas y el flujo de trabajo de los oficinistas. Se almacenarán
cada uno en una matriz de NxN y serán las entradas del problema.
\item \textbf{Función solución.} Comprueba, en cada paso, si el subconjunto
actual de candidatos elegidos forma una solución (no importa si es
óptima o no lo es). En nuestro caso una vez el conjunto de candidatos
esté vacío habremos llegado a la solución. 
\item \textbf{Función selección.} Informa cuál es el elemento más prometedor
para completar la solución. Éste no puede haber sido escogido con
anterioridad. Cada elemento es considerado una sola vez. Luego, puede
ser rechazado o aceptado. Para nuestro algoritmo escogemos el oficinista
con menor flujo de trabajo y le asignamos la habitación con mayor
distancia.
\item \textbf{Función de factibilidad.} Informa si a partir de un conjunto
se puede llegar a una solución. En este caso la función de selección
ya implementa la de factibilidad porque escogemos un elemento que
es factible.
\item \textbf{Función objetivo.} Es aquella que queremos maximizar o minimizar,
el núcleo del problema. En nuestro problema queremos asignar a oficinistas
con la máxima carga de trabajo a habitaciones con la mínima distancia. 
\end{itemize}

\section{Pseudocódigo que soluciona el problema}

\section{Explicación del funcionamiento del algoritmo sobre el ejemplo}

\section{Enunciado de un problema o caso real donde se pueda aplicar el algoritmo}

El algorítmo de Asignación Cuadrática explicado es utilizado actualmente es muchas empresas para optimizar la ganancia de beneficios gastando el mínimo número de recursos.

Algunos de los ejemplos que hemos visto más destacables y que muestran el caso de forma explícita son:

Diseño de centros comerciales donde se quiere que el público recorra
la menor cantidad de distancia para llegar a tiendas de mayor interés común. 

Diseño de circuitos eléctricos, en donde es de relevante importancia
dónde se ubican ciertas partes o chips con el fin de minimizar la
distancia entre ellos, ya que las conexiones son de alto costo.

Para la resolución de ambos problemas, podríamos aplicar exactamente los mismos pasos utilizados en nuestro algoritmo para obtener una solución adecuada.

\section{Cálculo del orden de eficiencia teórica del algoritmo }

\section{Instrucciones sobre como compilar y ejecutar el código de la práctica}

\textbf{Compilación:} Ejecutar make en su terminal.


\textbf{Ejecución:} 


1. Crear un archivo de texto con el siguiente formato.
\begin{itemize}
\item Numero de filas/columnas. 
\item Matriz de distancias a Habitaciones.
\item Matriz de flujo de trabajo de Oficinistas.
\end{itemize}
Formato similar al proporcionado en dat/matrices.dat


2. Ejecutar el programa que se guardará por defecto en la carpeta
bin del proyecto ./bin/prueba \textit{nombre\_archivo}
\end{document}


\grid
\grid
\grid
