\documentclass[12pt]{beamer}
\usetheme{Warsaw}
\usepackage[utf8]{inputenc}
\usepackage[spanish]{babel}
\usepackage{amsmath}
\usepackage{amsfonts}
\usepackage{amssymb}
\usepackage{graphicx}
\usepackage{color}
\definecolor{softgray}{rgb}{0.8,0.8,0.8}
\usepackage{listings} %Para codigo
\author{Daniel Bolaños Martínez, José María Borrás Serrano, Santiago De Diego De Diego, Fernando De la Hoz Moreno}
\title{Práctica 2: Algoritmos Divide y Vencerás}
%\setbeamercovered{transparent} 
%\setbeamertemplate{navigation symbols}{} 
%\logo{} 
\institute{ETSIIT} 
\date{} 

%\subject{} 
\begin{document}

\begin{frame}
\titlepage
\end{frame}

\begin{frame}
\tableofcontents
\end{frame}

\begin{frame}{Introducción}

\end{frame}

\begin{frame}[fragile]{Código Divide y Vencerás}
	\lstset{language=C++, breaklines=true, backgroundcolor=\color{softgray},keywordstyle=\color{blue},stringstyle=\color{orange},basicstyle=\footnotesize}
	\begin{lstlisting}
	
	\end{lstlisting}
\end{frame}

\begin{frame}[fragile]{Código secuencial}
	\lstset{language=C++, breaklines=true, extendedchars=true,backgroundcolor=\color{softgray}, keywordstyle=\color{blue},stringstyle=\color{orange},caption={Función unimodal}, basicstyle=\footnotesize}
	\begin{lstlisting}
	int unimodal_secuencial(vector<int> v)
	{
  		bool fin=false;
  		int indice=1;
  		while(!fin)
  		{
   	 		if(v.at(indice+1)<v.at(indice))
        			fin=true;
     		else
        		indice++;
  		}
 	 	return indice;
	}
	\end{lstlisting}
\end{frame}
\begin{frame}[fragile]
	\lstset{language=C++, breaklines=true, extendedchars=true, caption={Función main},backgroundcolor=\color{softgray},keywordstyle=\color{blue},stringstyle=\color{orange},basicstyle=\tiny}
	\begin{lstlisting}
int main(int argc, char* argv[])
{
  	vector<int> array;
  	int valor = -1;
  	int v_size = atoi(argv[1]);
  	array.resize(v_size);
    int p = 1 + rand() % (v_size-2);
    array.at(p) = v_size-1;
    for (int i=0; i<p; i++) 
        array.at(i)=i;
    for (int i=p+1; i<v_size; i++) 
        array.at(i)=v_size-1-i+p;

  	clock_t tantes;
  	clock_t tdespues;
  	tantes=clock();
  	valor = unimodal_secuencial(array);
 	tdespues=clock();
  	for(int i=0; i < v_size; i++)
     	cout << array.at(i) << endl;
    cout << "Maximo: " << array.at(valor) << endl;
	cout << v_size <<" "<< (double)(tdespues - tantes) / CLOCKS_PER_SEC << endl;
}
	\end{lstlisting}

\end{frame}

\begin{frame}{Comparación de la eficiencia}

\begin{block}{Eficiencia teórica}
\begin{figure}[H] 
\centering
%\includegraphics[angle=90,scale=0.4]{img/nombre.jpg} 
\caption{Pie de imagen} 
\label{etiqueta} 
\end{figure}
\end{block}

\begin{block}{Eficiencia empírica}

\end{block}

\begin{block}{Eficiencia híbrida}

\end{block}

\end{frame}

\begin{frame}{Conclusión}

\end{frame}

\end{document}
